\newif\ifvimbug
\vimbugfalse

\ifvimbug
\begin{document}
\fi

\exercise{Statistics Refresher}
 

\begin{questions}

%----------------------------------------------

\begin{question}{Expectation and Variance}{8}
Let $\Omega$ be a finite set and $P:\Omega\rightarrow\R$ a probability measure that (by definition) satisfies $P(\omega)\geq0$ for all $\omega\in\Omega$ and $\sum_{\omega\in\Omega}P(\omega)=1$. 
Let $f:\Omega\rightarrow\R$ be an arbitrary function on $\Omega$.

\textbf{1)} Write the definition of expectation and variance of $f$ and discuss if they are linear operators.

\textbf{2)} You are given a set of three dices $\{A,B,C\}$.
The following table describes the outcome of six rollouts for these dices, where each column shows the outcome of the respective dice. 
(Note: assume the dices are standard six-sided dices with values between 1-6)
\begin{equation*}
\begin{array}{r|cccccc}
    A & 4 & 4 & 2 & 4 & 1 & 1 \\
    \hline
    B & 3 & 6 & 3 & 3 & 4 & 3 \\
    \hline
    C & 5 & 5 & 2 & 1 & 1 & 1 
\end{array}
\end{equation*}
Estimate the expectation and the variance for each dice using unbiased estimators. (Show your computations).

\textbf{3)} According to the data, which of them is the ``most rigged''? Why?

\begin{answer}
\textbf{ANSWER}\\

1). 
From the description given we can imagine that the set $\Omega$ is a discrete set which for each element $\omega$ in the set $\Omega$ has the corresponding probability $P(\omega)$. we assume the function of $f$ is $f(\omega)$, so the expectation will be :
$$E(f(\omega)) = \sum_{\omega \in \Omega}f(\omega)P(\omega)$$
and the variance of $f(\omega)$ is :
$$Var(f(\omega)) = E(f(\omega)- E(f(\omega)))^2$$
Because all the element in the set $\Omega$ is discrete, which means the expectation calculation is linear, but the variance calculation consists of the square, which is nonlinear.

2).
For dices there are six number which are $1,2,3,4,5,6$, the probability of showing each number is equal. Means it's an unit distribution and $P = \frac{1}{6}$. \\
The expectation and variance of dice is :
$$E(dice) = \sum_{i = 1}^6 \frac{1}{6 }i = \frac{7}{2}$$
For dice A:
$$\overline{ X}_{A} = \sum_{i=1}^6 \frac{1}{6}x_i = \frac{4}{6}+\frac{4}{6}+\frac{2}{6}+\frac{4}{6}+ 2* \frac{1}{6} = \frac{8}{3}$$
$$S_A^2 =\frac{1}{n-1}\sum_{i=1}^6 (x_i-\overline{X}_A)^2 = \frac{1}{5}((\frac{4}{3})^2*3+ (\frac{2}{3})^2+ (\frac{5}{3})^2*2 )= 2.27$$
For dice B:
$$\overline{ X}_{B} = \sum_{i=1}^6 \frac{1}{6}x_i = \frac{3}{6}+\frac{6}{6}+\frac{3}{6}+\frac{3}{6}+ \frac{4}{6} + \frac{3}{6} = \frac{11}{3}$$
$$S_B^2 =\frac{1}{n-1}\sum_{i=1}^6 (x_i-\overline{X}_B)^2 = \frac{1}{5}((\frac{2}{3})^2*4+ (\frac{7}{3})^2+ (\frac{1}{3})^2) = 1.46$$

For dice C:
$$\overline{ X}_{C} = \sum_{i=1}^6 \frac{1}{6}x_i = \frac{5}{6}*2+\frac{2}{6}+\frac{1}{6}*3 = \frac{5}{3}$$
$$S_C^2 =\frac{1}{n-1}\sum_{i=1}^6 (x_i-\overline{X}_C)^2 = \frac{1}{5}((\frac{10}{3})^2*4+ (\frac{1}{3})^2+ (\frac{2}{3})^2*3) = 4.73$$

c).
Since the dice C has the most larger Variance and also the expectation of c with respect to the $E(dice)$ is also the largest. Therefore the dice c is most rigged.


\end{answer}

\end{question}
\newpage
%----------------------------------------------

\begin{question}{It is a Cold World}{7}
Consider the following three statements:
\\
a) A person with a cold has backpain $24\%$ of the time.
\\
b) $5\%$ of the world population has a cold.
\\
c) $12\%$ of those who do not have a cold, still have backpain.

\textbf{1)} Identify random variables from the statements above and define a unique symbol for each of them.\\
\textbf{2)} Define the domain of each random variable.\\
\textbf{3)} Represent the three statements above with your random variables.\\
\textbf{4)} If you suffer from backpain, what are the chances that you suffer from a cold? (Show all the intermediate steps.)

\begin{answer}
\textbf{ANSWER}\\
1)\\
"cold" represents the people get colded,\\
"backpain" represents the people has backpain,\\
2)\\
The domain of cold and backpain are the whole population. Means no matter if one person get cold or have backpain, they all belong to the same world.\\
3)\\
statement a) : $P(backpain | cold) = 24\%$ \\
statement b): $P(cold) = 5\%$ \\
statement c): $P(backpain) = 12\%$

4)\\
Calculation:\\
 $$P(cold | backpain) =\frac{P(backpain, cold)}{P(backpain)} = \frac{P(cold)*P(backpain | cold)}{P(backpain)} = \frac{0.24 * 0.05}{0.12} =  10\%$$

\end{answer}

\end{question}

\newpage
%----------------------------------------------

\begin{question}{Journey to THX1138}{10}
	After the success of the \href{http://rosetta.esa.int/}{Rosetta mission}, ESA decided to send a spaceship to rendezvous with the comet THX1138. 
	This spacecraft consists of four independent subsystems $A,B,C,D$. 
	Each subsystem has a probability of failing during the journey equal to $1/3$. 
	\\
	1) What is the probability of the spacecraft $S$ to be in working condition (i.e., all subsystems are operational at the same time) at the rendezvous?
	\\
	2) Given that the spacecraft $S$ is not operating properly, compute	analytically the probability that \textbf{only} subsystem $A$ has failed. 
	\\
	3) Instead of computing the probability analytically, do a simple simulation experiment and compare the result to the previous solution. 
	Include a snippet of your code. 
	\\
	4) An improved spacecraft version has been designed.
	The new spacecraft fails if the critical subsystem $A$ fails, or any two subsystems of the remaining $B,C,D$ fail. 
	What is the probability that \textbf{only} subsystem $A$ has failed, given that the spacecraft $S$ is failing? 
	
	
\begin{answer}
\textbf{ANSWER}\\
1)\\
SInce all the subsystems work well which means the probability of all the subsystem should be $\frac{3}{4}$ So the total probability should be :
$$P_{well} = (\frac{2}{3})^4 = \frac{16}{81}$$
2)\\
The probability that all the subsystem work properly is:
$$P(well) = (\frac{2}{3})^4 = \frac{16}{18}$$
So the probability that the spaceship doesn't work properly is $P(spaceship\_fail) = 1-P(spaceship\_well) = \frac{65}{81}$
The propability that only subsystem isn't working properly is :
$$P(A\_fail) = P_A(fail) * P_B(well) * P_C(well) * P_D(well) = \frac{1}{3}* (\frac{2}{3})^3 = \frac{8}{81}$$
Therefore we need to calculate the probability that if the whole spaceship doesn't work properly and the reason is only A is broken. which is $P(A\_fail | spaceship\_fail)$

$$P(A\_fail | spaceship\_fail) = \frac{P(A\_fail)}{P(spaceship\_fail)} = \frac{8}{65}$$

3)\\
\begin{code}
import numpy as np \\
time = 100000\\
system = []\\
ans = []\\
A = np.random.binomial(1,1/3, time)\\
B = np.random.binomial(1,1/3, time)\\
C = np.random.binomial(1,1/3, time)\\
D = np.random.binomial(1,1/3, time)\\
for i in range(len(A)):\\
    if A[i]==1 or B[i]==1 or C[i]==1 or D[i] ==1:\\
        system.append(1)\\
    if A[i]==1 and B[i]!=1 and C[i]!=1 and D[i] !=1:\\
        ans.append(1)\\
p = sum(ans) / sum(system)\\
\end{code}

4)\\
$$P(spacecraft\_fail) = \frac{1}{3}*(\frac{2}{3})^2 + \frac{2}{3}* 3*(\frac{1}{3})^2*\frac{2}{3}= \frac{8}{9}$$
$$P_A(fail) = \frac{1}{3}*(\frac{2}{3})^3 = \frac{8}{81}$$
so the final result will be 

$$P(A\_fail | spacecraft\_fail) =\frac{1}{9} $$

\end{answer}
	
\end{question}


\end{questions}
